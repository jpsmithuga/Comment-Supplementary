% Options for packages loaded elsewhere
\PassOptionsToPackage{unicode}{hyperref}
\PassOptionsToPackage{hyphens}{url}
%
\documentclass[
]{article}
\usepackage{amsmath,amssymb}
\usepackage{lmodern}
\usepackage{iftex}
\ifPDFTeX
  \usepackage[T1]{fontenc}
  \usepackage[utf8]{inputenc}
  \usepackage{textcomp} % provide euro and other symbols
\else % if luatex or xetex
  \usepackage{unicode-math}
  \defaultfontfeatures{Scale=MatchLowercase}
  \defaultfontfeatures[\rmfamily]{Ligatures=TeX,Scale=1}
\fi
% Use upquote if available, for straight quotes in verbatim environments
\IfFileExists{upquote.sty}{\usepackage{upquote}}{}
\IfFileExists{microtype.sty}{% use microtype if available
  \usepackage[]{microtype}
  \UseMicrotypeSet[protrusion]{basicmath} % disable protrusion for tt fonts
}{}
\makeatletter
\@ifundefined{KOMAClassName}{% if non-KOMA class
  \IfFileExists{parskip.sty}{%
    \usepackage{parskip}
  }{% else
    \setlength{\parindent}{0pt}
    \setlength{\parskip}{6pt plus 2pt minus 1pt}}
}{% if KOMA class
  \KOMAoptions{parskip=half}}
\makeatother
\usepackage{xcolor}
\IfFileExists{xurl.sty}{\usepackage{xurl}}{} % add URL line breaks if available
\IfFileExists{bookmark.sty}{\usepackage{bookmark}}{\usepackage{hyperref}}
\hypersetup{
  pdftitle={Supplementary Materials: It matters who is spreading monkeypox},
  pdfauthor={Jonathan Smith, PhD, MPH},
  hidelinks,
  pdfcreator={LaTeX via pandoc}}
\urlstyle{same} % disable monospaced font for URLs
\usepackage[margin=1in]{geometry}
\usepackage{color}
\usepackage{fancyvrb}
\newcommand{\VerbBar}{|}
\newcommand{\VERB}{\Verb[commandchars=\\\{\}]}
\DefineVerbatimEnvironment{Highlighting}{Verbatim}{commandchars=\\\{\}}
% Add ',fontsize=\small' for more characters per line
\usepackage{framed}
\definecolor{shadecolor}{RGB}{248,248,248}
\newenvironment{Shaded}{\begin{snugshade}}{\end{snugshade}}
\newcommand{\AlertTok}[1]{\textcolor[rgb]{0.94,0.16,0.16}{#1}}
\newcommand{\AnnotationTok}[1]{\textcolor[rgb]{0.56,0.35,0.01}{\textbf{\textit{#1}}}}
\newcommand{\AttributeTok}[1]{\textcolor[rgb]{0.77,0.63,0.00}{#1}}
\newcommand{\BaseNTok}[1]{\textcolor[rgb]{0.00,0.00,0.81}{#1}}
\newcommand{\BuiltInTok}[1]{#1}
\newcommand{\CharTok}[1]{\textcolor[rgb]{0.31,0.60,0.02}{#1}}
\newcommand{\CommentTok}[1]{\textcolor[rgb]{0.56,0.35,0.01}{\textit{#1}}}
\newcommand{\CommentVarTok}[1]{\textcolor[rgb]{0.56,0.35,0.01}{\textbf{\textit{#1}}}}
\newcommand{\ConstantTok}[1]{\textcolor[rgb]{0.00,0.00,0.00}{#1}}
\newcommand{\ControlFlowTok}[1]{\textcolor[rgb]{0.13,0.29,0.53}{\textbf{#1}}}
\newcommand{\DataTypeTok}[1]{\textcolor[rgb]{0.13,0.29,0.53}{#1}}
\newcommand{\DecValTok}[1]{\textcolor[rgb]{0.00,0.00,0.81}{#1}}
\newcommand{\DocumentationTok}[1]{\textcolor[rgb]{0.56,0.35,0.01}{\textbf{\textit{#1}}}}
\newcommand{\ErrorTok}[1]{\textcolor[rgb]{0.64,0.00,0.00}{\textbf{#1}}}
\newcommand{\ExtensionTok}[1]{#1}
\newcommand{\FloatTok}[1]{\textcolor[rgb]{0.00,0.00,0.81}{#1}}
\newcommand{\FunctionTok}[1]{\textcolor[rgb]{0.00,0.00,0.00}{#1}}
\newcommand{\ImportTok}[1]{#1}
\newcommand{\InformationTok}[1]{\textcolor[rgb]{0.56,0.35,0.01}{\textbf{\textit{#1}}}}
\newcommand{\KeywordTok}[1]{\textcolor[rgb]{0.13,0.29,0.53}{\textbf{#1}}}
\newcommand{\NormalTok}[1]{#1}
\newcommand{\OperatorTok}[1]{\textcolor[rgb]{0.81,0.36,0.00}{\textbf{#1}}}
\newcommand{\OtherTok}[1]{\textcolor[rgb]{0.56,0.35,0.01}{#1}}
\newcommand{\PreprocessorTok}[1]{\textcolor[rgb]{0.56,0.35,0.01}{\textit{#1}}}
\newcommand{\RegionMarkerTok}[1]{#1}
\newcommand{\SpecialCharTok}[1]{\textcolor[rgb]{0.00,0.00,0.00}{#1}}
\newcommand{\SpecialStringTok}[1]{\textcolor[rgb]{0.31,0.60,0.02}{#1}}
\newcommand{\StringTok}[1]{\textcolor[rgb]{0.31,0.60,0.02}{#1}}
\newcommand{\VariableTok}[1]{\textcolor[rgb]{0.00,0.00,0.00}{#1}}
\newcommand{\VerbatimStringTok}[1]{\textcolor[rgb]{0.31,0.60,0.02}{#1}}
\newcommand{\WarningTok}[1]{\textcolor[rgb]{0.56,0.35,0.01}{\textbf{\textit{#1}}}}
\usepackage{graphicx}
\makeatletter
\def\maxwidth{\ifdim\Gin@nat@width>\linewidth\linewidth\else\Gin@nat@width\fi}
\def\maxheight{\ifdim\Gin@nat@height>\textheight\textheight\else\Gin@nat@height\fi}
\makeatother
% Scale images if necessary, so that they will not overflow the page
% margins by default, and it is still possible to overwrite the defaults
% using explicit options in \includegraphics[width, height, ...]{}
\setkeys{Gin}{width=\maxwidth,height=\maxheight,keepaspectratio}
% Set default figure placement to htbp
\makeatletter
\def\fps@figure{htbp}
\makeatother
\setlength{\emergencystretch}{3em} % prevent overfull lines
\providecommand{\tightlist}{%
  \setlength{\itemsep}{0pt}\setlength{\parskip}{0pt}}
\setcounter{secnumdepth}{-\maxdimen} % remove section numbering
\ifLuaTeX
  \usepackage{selnolig}  % disable illegal ligatures
\fi

\title{Supplementary Materials: It matters who is spreading monkeypox}
\author{Jonathan Smith, PhD, MPH}
\date{2022-07-24}

\begin{document}
\maketitle

\begin{center}\rule{0.5\linewidth}{0.5pt}\end{center}

\begin{center}\rule{0.5\linewidth}{0.5pt}\end{center}

Written in R version 4.1.2.

\begin{center}\rule{0.5\linewidth}{0.5pt}\end{center}

\hypertarget{background}{%
\subsection{Background}\label{background}}

We assume the number of secondary cases produced from each infectious
case is distributed according to a negative binomial distribution with
mean \(R_0\) (the basic reproductive number) and dispersion parameter
\(k\). The parameter \(k\) quantifies the degree of individual
heterogeneity by measuring overdispersion in the distribution (e.g.,
higher than expected variation). For a given \(R<1\), smaller values of
\(k\) (\(k<<1\)) suggest increased heterogeneity in the number of
secondary cases between individual index cases. As \(k\) increases,
transmission becomes more uniform and centered around the mean, \(R_0\).

\hypertarget{proportion-of-secondary-cases-attributed-to-the-proportion-of-infectious-cases}{%
\subsection{Proportion of secondary cases attributed to the proportion
of infectious
cases}\label{proportion-of-secondary-cases-attributed-to-the-proportion-of-infectious-cases}}

Lloyd-Smith \emph{et al.} (L-S)
(\href{https://static-content.springer.com/esm/art\%3A10.1038\%2Fnature04153/MediaObjects/41586_2005_BFnature04153_MOESM1_ESM.pdf}{supplementary
information 2.2.5}) propose a distribution for describing transmission
based from the distribution of the individual reproduction number
\(\nu\):
\[ F_\mathrm{trans}(x) = \frac{1}{R_0} \int_0^x u\,  f_\nu (u) \, du \]
This gives the cumulative distribution function (CDF) in terms of the
individual reproduction number density \(f_\nu\), as the proportion of
all transmission due to infectious individuals with reproduction number
\(\nu <x\).

In the specific case when \(\nu\) is gamma distributed with shape
parameter \(k>0\) and rate parameter \(k/R_0\), where \(R_0\) is the
mean (expected) individual reproduction number, we have
\[ f_\nu(u) = \frac{k^k}{R_0^k \, \Gamma(k)} u^{k-1} e^{-ku/R_0} \] It
follows that
\[ \begin{eqnarray*} f_\mathrm{trans}(x) =  F'_\mathrm{trans}(x) 
&=& \frac{1}{R_0} x f_\nu(x) 
\\
&=& \frac{k^k}{R_0^{k+1} \, \Gamma(k)} x^k e^{-kx/R_0} 
\\
&=& \frac{k^{k+1}}{R_0^{k+1} \, \Gamma(k+1)} x^k e^{-kx/R_0} 
\\
&\sim & \mathrm{Gamma}(k+1, k/R_0) 
\end{eqnarray*}\]

The rate parameter remains the same and the shape parameter increases by
1.

To calculate \(t_p\), the expected proportion of transmission due to the
most infectious \(100p\)\% of cases, we first find the \((1-p)\)th
centile of the individual reproduction number,
\(x_p = F^{-1}_\nu(1-p)\), then calculate
\(t_p = 1-F_\mathrm{trans}(x_p)\). As both random variables are gamma
distributed, this is implemented in R via the qgamma (\(F^{-1}\)) and
pgamma (\(F\)) functions.

\begin{Shaded}
\begin{Highlighting}[]
\NormalTok{propinfection }\OtherTok{\textless{}{-}} \ControlFlowTok{function}\NormalTok{(R, k, prop)\{}
\NormalTok{  xp }\OtherTok{\textless{}{-}} \FunctionTok{qgamma}\NormalTok{(}\DecValTok{1} \SpecialCharTok{{-}}\NormalTok{ prop, }\AttributeTok{shape =}\NormalTok{ k, }\AttributeTok{rate =}\NormalTok{ k}\SpecialCharTok{/}\NormalTok{R) }
\NormalTok{  tp }\OtherTok{\textless{}{-}} \DecValTok{1} \SpecialCharTok{{-}} \FunctionTok{pgamma}\NormalTok{(xp, }\AttributeTok{shape =}\NormalTok{ k}\SpecialCharTok{+}\DecValTok{1}\NormalTok{, }\AttributeTok{rate =}\NormalTok{ k}\SpecialCharTok{/}\NormalTok{R) }
  \FunctionTok{return}\NormalTok{(tp) }
\NormalTok{\}}
\end{Highlighting}
\end{Shaded}

\hypertarget{the-relationship-between-p-and-t_p}{%
\subsection{\texorpdfstring{The relationship between \(p\) and
\(t_p\)}{The relationship between p and t\_p}}\label{the-relationship-between-p-and-t_p}}

Parameterizing by \(x_p\), and noting that \(p=1-F_\nu(x_p)\), the first
derivative is \[ \frac{d t_p}{dp} = \frac{dt_p/dx_p}{dp/dx_p} 
= \frac{-F'_\mathrm{trans}(x_p)}{-F'_\nu(x_p)} = 
\frac{R_0^{-1} x_p f_\nu(x_p)}{f_\nu(x_p)} = \frac{x_p}{R_0} > 0\] which
confirms that \(t_p\) is an increasing function of \(p\). When
\(x_p=0\), \((p, t_p) = (1,1)\) at which the point the derivative is
zero (horizontal tangent). Similarly, when \(x_p=+\infty\),
\((p, t_p) = (0,0)\) at which the point the derivative is infinite
(vertical tangent).

The second derivative is negative implying concavity of the graph of
\(t_p\) as a function of \(p\). \[ \frac{d^2 t_p}{dp^2} 
= \frac{d}{dx_p} \left(\frac{d t_p}{dp}\right) \times \frac{d x_p}{dp} 
= \frac{1}{R_0} \times \frac{1}{dp/dx_p} = - \frac{1}{R_0 f_\nu(x_p)} <0\]
These properties can be seen in Figure 1A:

\begin{Shaded}
\begin{Highlighting}[]
\NormalTok{xx }\OtherTok{\textless{}{-}} \FunctionTok{seq}\NormalTok{(}\DecValTok{0}\NormalTok{, }\DecValTok{1}\NormalTok{, }\FloatTok{0.001}\NormalTok{)}
\NormalTok{k }\OtherTok{\textless{}{-}} \FunctionTok{c}\NormalTok{(}\FloatTok{0.1}\NormalTok{, }\FloatTok{0.5}\NormalTok{, }\DecValTok{2}\NormalTok{, }\DecValTok{100000}\NormalTok{) }\CommentTok{\# 100000 {-}\textgreater{} \textasciitilde{}infty (homogenous transmission)}
\NormalTok{plotdata }\OtherTok{\textless{}{-}} \FunctionTok{cbind}\NormalTok{(xx, }
                  \FunctionTok{propinfection}\NormalTok{(}\AttributeTok{R =} \DecValTok{3}\NormalTok{, }\AttributeTok{k =}\NormalTok{ k[}\DecValTok{1}\NormalTok{], xx),}
                  \FunctionTok{propinfection}\NormalTok{(}\AttributeTok{R =} \DecValTok{3}\NormalTok{, }\AttributeTok{k =}\NormalTok{ k[}\DecValTok{2}\NormalTok{], xx),}
                  \FunctionTok{propinfection}\NormalTok{(}\AttributeTok{R =} \DecValTok{3}\NormalTok{, }\AttributeTok{k =}\NormalTok{ k[}\DecValTok{3}\NormalTok{], xx),}
                  \FunctionTok{propinfection}\NormalTok{(}\AttributeTok{R =} \DecValTok{3}\NormalTok{, }\AttributeTok{k =}\NormalTok{ k[}\DecValTok{4}\NormalTok{], xx))}

                            
\FunctionTok{plot}\NormalTok{(plotdata[,}\DecValTok{1}\NormalTok{], plotdata[,}\DecValTok{2}\NormalTok{], }\AttributeTok{type =} \StringTok{"l"}\NormalTok{, }\AttributeTok{bty =} \StringTok{"n"}\NormalTok{,}
     \AttributeTok{xlab =} \StringTok{"Proportion of Infectious Cases"}\NormalTok{,}
     \AttributeTok{ylab =} \StringTok{"Expected Proportion of Secondary Cases"}\NormalTok{,}
     \AttributeTok{lwd =} \DecValTok{2}\NormalTok{, }\AttributeTok{lty =} \DecValTok{4}\NormalTok{)}
\FunctionTok{lines}\NormalTok{(plotdata[,}\DecValTok{1}\NormalTok{], plotdata[,}\DecValTok{3}\NormalTok{], }\AttributeTok{lty =} \DecValTok{3}\NormalTok{, }\AttributeTok{lwd =} \DecValTok{2}\NormalTok{)}
\FunctionTok{lines}\NormalTok{(plotdata[,}\DecValTok{1}\NormalTok{], plotdata[,}\DecValTok{4}\NormalTok{], }\AttributeTok{lty =} \DecValTok{2}\NormalTok{, }\AttributeTok{lwd =} \DecValTok{2}\NormalTok{)}
\FunctionTok{lines}\NormalTok{(plotdata[,}\DecValTok{1}\NormalTok{], plotdata[,}\DecValTok{5}\NormalTok{], }\AttributeTok{lty =} \DecValTok{1}\NormalTok{, }\AttributeTok{lwd =} \DecValTok{2}\NormalTok{)}
\FunctionTok{legend}\NormalTok{(}\AttributeTok{x =} \FloatTok{0.6}\NormalTok{, }\AttributeTok{y =} \FloatTok{0.35}\NormalTok{, }\FunctionTok{c}\NormalTok{(}\StringTok{"k = 0.1"}\NormalTok{, }\StringTok{"k = 0.5"}\NormalTok{, }\StringTok{"k = 2.0"}\NormalTok{, }\StringTok{"No Variation"}\NormalTok{), }
       \AttributeTok{lty =} \DecValTok{4}\SpecialCharTok{:}\DecValTok{1}\NormalTok{, }\AttributeTok{bty =} \StringTok{"n"}\NormalTok{, }\AttributeTok{lwd =} \DecValTok{2}\NormalTok{)}
\FunctionTok{mtext}\NormalTok{(}\FunctionTok{bquote}\NormalTok{(R[}\DecValTok{0}\NormalTok{]}\SpecialCharTok{\textasciitilde{}}\StringTok{"= 3.0"}\NormalTok{), }\AttributeTok{adj =} \DecValTok{1}\NormalTok{)}
\end{Highlighting}
\end{Shaded}

\includegraphics{smith_response_LancetID_files/figure-latex/unnamed-chunk-2-1.pdf}

\hypertarget{final-outbreak-size}{%
\subsection{Final Outbreak Size}\label{final-outbreak-size}}

Let \(Y\) represent the sum of all cases in a transmission chain
(including the index case), resulting in a final outbreak size (Dwass,
1969). Using branching process theory, the distribution of the total
number of secondary infections in a single transmission chain (including
the index case at generation) can be defined by the probability
generating function (pgf) yielding the implicit relationship (Yan 2008):

\[G_Y(s) = sG_Z((G_Y(s))\] We can treat this as
\(G_Y(s) = \sum_{y=1}^\infty q_ys^y\), where \(q_y = P(Y=y)\), allowing
us to extract the probability (Becker 1974; Blumberg 2014):

\[
P(Y=y)=\frac{1}{y!}\frac{d_yG_Y(s)}{ds^y}\Bigg|_{s=0}
\] Where \(Y=1, 2, 3\dots\)

Plugging in the negative binomial distribution yields (see Smith 2022
(\href{}{supplementary information, section 4}) and Blumberg 2012) \[
P(Y=y)=\bigg(\frac{n}{y}\bigg)\frac{\Gamma(ky+y-n)}{\Gamma(ky)(y-n)!}\frac{(\frac{R_0}{k})^{y-n}}{(1+\frac{R_0}{k})^{ky+y-n}}
\] Where \(n\) represents the number of index cases initiating the
outbreak and \(Y=1, 2, 3\dots\).

The this is implemented in R software by exponentiating the
log-likelihood (for computational ease):

\begin{Shaded}
\begin{Highlighting}[]
\NormalTok{outbreakprob }\OtherTok{\textless{}{-}} \ControlFlowTok{function}\NormalTok{(y, R, k)\{}
\NormalTok{  l }\OtherTok{\textless{}{-}} \FunctionTok{lgamma}\NormalTok{(k }\SpecialCharTok{*}\NormalTok{ y }\SpecialCharTok{+}\NormalTok{ (y }\SpecialCharTok{{-}} \DecValTok{1}\NormalTok{)) }\SpecialCharTok{{-}}\NormalTok{ (}\FunctionTok{lgamma}\NormalTok{(k }\SpecialCharTok{*}\NormalTok{ y) }\SpecialCharTok{+} \FunctionTok{lgamma}\NormalTok{(y }\SpecialCharTok{+} \DecValTok{1}\NormalTok{)) }\SpecialCharTok{+}\NormalTok{ (y }\SpecialCharTok{{-}} \DecValTok{1}\NormalTok{) }\SpecialCharTok{*} \FunctionTok{log}\NormalTok{(R }\SpecialCharTok{/}\NormalTok{ k) }\SpecialCharTok{{-}}\NormalTok{ (k }\SpecialCharTok{*}\NormalTok{ y }\SpecialCharTok{+}\NormalTok{ (y }\SpecialCharTok{{-}} \DecValTok{1}\NormalTok{)) }\SpecialCharTok{*} \FunctionTok{log}\NormalTok{(}\DecValTok{1} \SpecialCharTok{+}\NormalTok{ R }\SpecialCharTok{/}\NormalTok{ k)}
  \FunctionTok{return}\NormalTok{(}\FunctionTok{exp}\NormalTok{(l))}
\NormalTok{\}}
\end{Highlighting}
\end{Shaded}

Figure 1B is generated with the following code:

\begin{Shaded}
\begin{Highlighting}[]
\NormalTok{R }\OtherTok{\textless{}{-}} \DecValTok{3}
\NormalTok{maxoutbreak }\OtherTok{\textless{}{-}} \DecValTok{100}
\NormalTok{outbrk\_prob1 }\OtherTok{\textless{}{-}} \FunctionTok{outbreakprob}\NormalTok{(}\DecValTok{1}\SpecialCharTok{:}\NormalTok{maxoutbreak, R, k[}\DecValTok{1}\NormalTok{])}
\NormalTok{outbrk\_prob2 }\OtherTok{\textless{}{-}} \FunctionTok{outbreakprob}\NormalTok{(}\DecValTok{1}\SpecialCharTok{:}\NormalTok{maxoutbreak, R, k[}\DecValTok{2}\NormalTok{])}
\NormalTok{outbrk\_prob3 }\OtherTok{\textless{}{-}} \FunctionTok{outbreakprob}\NormalTok{(}\DecValTok{1}\SpecialCharTok{:}\NormalTok{maxoutbreak, R, k[}\DecValTok{3}\NormalTok{])}
\NormalTok{outbrk\_prob4 }\OtherTok{\textless{}{-}} \FunctionTok{outbreakprob}\NormalTok{(}\DecValTok{1}\SpecialCharTok{:}\NormalTok{maxoutbreak, R, k[}\DecValTok{4}\NormalTok{])}

\NormalTok{xx }\OtherTok{\textless{}{-}} \FunctionTok{seq}\NormalTok{(}\DecValTok{0}\NormalTok{, maxoutbreak, }\AttributeTok{by =} \FloatTok{0.01}\NormalTok{)}
\NormalTok{maxrange }\OtherTok{\textless{}{-}} \FunctionTok{seq}\NormalTok{(}\DecValTok{1}\NormalTok{, maxoutbreak, }\DecValTok{1}\NormalTok{)}

\FunctionTok{plot}\NormalTok{(}\FunctionTok{log}\NormalTok{(}\FunctionTok{range}\NormalTok{(}\DecValTok{1}\NormalTok{, maxoutbreak)), }\FunctionTok{range}\NormalTok{(}\SpecialCharTok{{-}}\DecValTok{11}\NormalTok{,}\DecValTok{0}\NormalTok{), }\AttributeTok{type =} \StringTok{\textquotesingle{}n\textquotesingle{}}\NormalTok{, }\AttributeTok{xlab =} \StringTok{\textquotesingle{}\textquotesingle{}}\NormalTok{, }\AttributeTok{ylab =} \StringTok{\textquotesingle{}\textquotesingle{}}\NormalTok{, }\AttributeTok{axes =}\NormalTok{ F)}
  \FunctionTok{points}\NormalTok{(}\FunctionTok{log}\NormalTok{(maxrange), }\FunctionTok{log}\NormalTok{(outbrk\_prob1), }\AttributeTok{col =} \DecValTok{1}\NormalTok{, }\AttributeTok{pch=}\DecValTok{2}\NormalTok{, }\AttributeTok{cex=}\DecValTok{1}\NormalTok{)}
  \FunctionTok{points}\NormalTok{(}\FunctionTok{log}\NormalTok{(maxrange), }\FunctionTok{log}\NormalTok{(outbrk\_prob2), }\AttributeTok{col =} \DecValTok{1}\NormalTok{, }\AttributeTok{pch=}\DecValTok{3}\NormalTok{, }\AttributeTok{cex=}\DecValTok{1}\NormalTok{)}
  \FunctionTok{points}\NormalTok{(}\FunctionTok{log}\NormalTok{(maxrange), }\FunctionTok{log}\NormalTok{(outbrk\_prob3), }\AttributeTok{col =} \DecValTok{1}\NormalTok{, }\AttributeTok{pch=}\DecValTok{4}\NormalTok{, }\AttributeTok{cex=}\DecValTok{1}\NormalTok{)}
  \FunctionTok{points}\NormalTok{(}\FunctionTok{log}\NormalTok{(maxrange), }\FunctionTok{log}\NormalTok{(outbrk\_prob4), }\AttributeTok{col =} \DecValTok{1}\NormalTok{, }\AttributeTok{pch=}\DecValTok{19}\NormalTok{, }\AttributeTok{cex=}\DecValTok{1}\NormalTok{)}
  \FunctionTok{axis}\NormalTok{(}\AttributeTok{side=}\DecValTok{1}\NormalTok{, }\AttributeTok{at=}\FunctionTok{log}\NormalTok{(}\FunctionTok{c}\NormalTok{(}\DecValTok{1}\NormalTok{, }\DecValTok{5}\NormalTok{, }\FunctionTok{seq}\NormalTok{(}\DecValTok{10}\NormalTok{, maxoutbreak }\SpecialCharTok{+} \DecValTok{10}\NormalTok{, }\AttributeTok{b =} \DecValTok{10}\NormalTok{))), }\FunctionTok{c}\NormalTok{(}\DecValTok{1}\NormalTok{, }\DecValTok{5}\NormalTok{, }\FunctionTok{seq}\NormalTok{(}\DecValTok{10}\NormalTok{, maxoutbreak }\SpecialCharTok{+} \DecValTok{10}\NormalTok{, }\AttributeTok{b =} \DecValTok{10}\NormalTok{)))}
  \FunctionTok{axis}\NormalTok{(}\AttributeTok{side=}\DecValTok{2}\NormalTok{, }\AttributeTok{at=}\FunctionTok{log}\NormalTok{(}\FunctionTok{c}\NormalTok{(}\DecValTok{1}\SpecialCharTok{/}\DecValTok{100000}\NormalTok{, }\DecValTok{1}\SpecialCharTok{/}\DecValTok{10000}\NormalTok{, }\DecValTok{1}\SpecialCharTok{/}\DecValTok{1000}\NormalTok{, }\DecValTok{1}\SpecialCharTok{/}\DecValTok{100}\NormalTok{, }\DecValTok{1}\SpecialCharTok{/}\DecValTok{10}\NormalTok{, }\DecValTok{1}\SpecialCharTok{/}\DecValTok{2}\NormalTok{, }\DecValTok{1}\NormalTok{)), }\FunctionTok{c}\NormalTok{(}\DecValTok{1}\SpecialCharTok{/}\DecValTok{100000}\NormalTok{, }\DecValTok{1}\SpecialCharTok{/}\DecValTok{10000}\NormalTok{, }\DecValTok{1}\SpecialCharTok{/}\DecValTok{1000}\NormalTok{, }\DecValTok{1}\SpecialCharTok{/}\DecValTok{100}\NormalTok{, }\DecValTok{1}\SpecialCharTok{/}\DecValTok{10}\NormalTok{, }\DecValTok{1}\SpecialCharTok{/}\DecValTok{2}\NormalTok{, }\DecValTok{1}\NormalTok{))}
  \FunctionTok{mtext}\NormalTok{(}\AttributeTok{side=}\DecValTok{1}\NormalTok{, }\StringTok{\textquotesingle{}Final Outbreak Size\textquotesingle{}}\NormalTok{, }\AttributeTok{padj=}\DecValTok{4}\NormalTok{)}
  \FunctionTok{mtext}\NormalTok{(}\AttributeTok{side=}\DecValTok{2}\NormalTok{, }\StringTok{\textquotesingle{}Probability\textquotesingle{}}\NormalTok{, }\AttributeTok{padj=}\SpecialCharTok{{-}}\DecValTok{4}\NormalTok{)}
  \FunctionTok{legend}\NormalTok{(}\StringTok{\textquotesingle{}topright\textquotesingle{}}\NormalTok{, }\FunctionTok{c}\NormalTok{(}\StringTok{"k = 0.1"}\NormalTok{, }\StringTok{"k = 0.5"}\NormalTok{, }\StringTok{"k = 2.0"}\NormalTok{, }\StringTok{"No Variation"}\NormalTok{), }
         \AttributeTok{pch=}\FunctionTok{c}\NormalTok{(}\DecValTok{2}\SpecialCharTok{:}\DecValTok{4}\NormalTok{, }\DecValTok{19}\NormalTok{), }\AttributeTok{cex=}\DecValTok{1}\NormalTok{, }\AttributeTok{bty=}\StringTok{"n"}\NormalTok{)}
  \FunctionTok{mtext}\NormalTok{(}\FunctionTok{bquote}\NormalTok{(R[}\DecValTok{0}\NormalTok{]}\SpecialCharTok{\textasciitilde{}}\StringTok{"= 3.0"}\NormalTok{), }\AttributeTok{adj =} \DecValTok{1}\NormalTok{)}
\end{Highlighting}
\end{Shaded}

\includegraphics{smith_response_LancetID_files/figure-latex/unnamed-chunk-4-1.pdf}

\hypertarget{thanks}{%
\subsection{Thanks}\label{thanks}}

Special thanks to Dr.~Andrew Hill for providing mathematical oversight
into this analysis.

\hypertarget{references}{%
\subsection{References}\label{references}}

\begin{enumerate}
\def\labelenumi{\arabic{enumi}.}
\item
  Lloyd-Smith, JO, \emph{et al}. Superspreading and the effect of
  individual variation on disease emergence, Nature 438, 355--359 (2005)
\item
  Yan, P. Distribution Theory, Stochastic Processes and Infectious
  Disease Modelling, pages 229. Springer Berlin Heidelberg, Berlin,
  Heidelberg, 2008.
\item
  Becker, N. On parametric estimation for mortal branching processes.
  Biometrika, 61(3):393, 1974.
\item
  Dwass, M. The total progeny in a branching process and a related
  random walk. Journal of Applied Probability, 6(3):682, 1969.
\item
  Blumberg S and Lloyd-Smith JO. Inference of r0 and transmission
  heterogeneity from the size distribution of stuttering chains. PLoS
  Comput Biol., 5(9):393, 2013.
\item
  Smith, JP, \emph{et al}. A cluster-based method to quantify individual
  heterogeneity in tuberculosis transmission. Epidemiology 33, 217--227.
  2022
\end{enumerate}

\end{document}
